\documentclass[12pt]{article}
\usepackage[spanish]{babel}
\usepackage[letterpaper,top=2cm,bottom=2cm,left=3cm,right=3cm,marginparwidth=1.75cm]{geometry}
\usepackage{tabularx}
\usepackage{fancyvrb}
\usepackage{graphicx}
\usepackage{setspace}
\usepackage{ragged2e}
\usepackage[T1]{fontenc}
\renewcommand*\familydefault{\sfdefault}
\usepackage{librefranklin}

\setlength{\parindent}{0pt}
\onehalfspacing

\begin{document}

\begin{titlepage}

    \centering
    \vspace*{1 cm}
    \Huge
    \textbf{Carrera de Coches}

    \vspace{0.5 cm}
    \Large
    Programación de Servicios y Procesos

    \vspace{5.5 cm}
    \textbf{Adrián Condines Celada}

    \vspace{0.8 cm}    
    \normalsize
    Aula Estudio\\

    \vspace{0.8 cm}
    2º Ciclo Superior - Desarrollo de Aplicaciones Multiplataforma\\

    \vspace{0.8 cm}
    Curso 2025 - 2026

\end{titlepage}

\tableofcontents

\newpage

\section{Introducción}

\justify
Esta es una aplicación Java que simula una competición entre seis vehículos mediante programación multihilo e interfaz gráfica JavaFX. 

\justify
La aplicación permite observar en tiempo real el progreso de cada coche mediante barras de progreso y una clasificación dinámica.

\section{Stack Tecnológico}

\subsection{Java}

\begin{figure}[h]

    \centering
    \includegraphics[width=3.5cm]{images/java_icon.png}

\end{figure}

\justify
El lenguaje de programación utililazo para desarrollar la lógica de la aplicación es Java, debido a su robustez y capacidad para manejar múltiples hilos de ejecución.

\subsection{JavaFX}

\begin{figure}[h]

    \centering
    \includegraphics[width=5cm]{images/javafx_icon.png}

\end{figure}

\justify
JavaFX se ha utilizado como tecnología para desarrollar la interfaz gráfica de la aplicación, proporcionando una manera visual de ver el resultado del programa.

\newpage

\subsection{CSS}

\begin{figure}[h]

    \centering
    \includegraphics[width=3.5cm]{images/css_icon.png}

\end{figure}

\justify
CSS se ha utilizado para dar estilo a la interfaz gráfica desarrollada con JavaFX, mejorando la experiencia del usuario.

\subsection{Visual Studio Code}

\begin{figure}[h]

    \centering
    \includegraphics[width=7cm]{images/vscode_icon.png}

\end{figure}

\justify
Como IDE se ha utilizado Visual Studio Code, que gracias a sus extensiones permite un desarrollo eficiente en Java y JavaFX.

\section{Requisitos Funcionales}

\justify
La aplicación cumple con los siguientes requisitos funcionales.

\justify
\textbf{RF-1}: Inciciación de la carrera al pulsar un botón.

\justify
\textbf{RF-2}: Simulación del movimiento de cada coche de manera aleatoria hasta que llegan a la meta.

\justify
\textbf{RF-3}: Barras de progreso que simulan la distancia recorrida por cada coche.

\justify
\textbf{RF-4}: Mostrar la clasificación final de los coches por orden de llegada.

\justify
\textbf{RF-5}: Posibilidad de reiniciar la carrera para una nueva simulación.

\section{Requisitos No Funcionales}

\justify
La aplicación cumple con los siguientes requisitos no funcionales.

\justify
\textbf{RNF-1}: Manejo correcto de hilos para evitar bloqueos.

\justify
\textbf{RNF-2}: Compatibilidad con todos los sistemas operativos, ya que está desarrollada en Java.

\justify
\textbf{RNF-3}: Interfaz sencilla e intuitiva para el usuario.

\section{Casos de Uso}

\justify
A continuación se describen los casos de uso principales de la aplicación:

\begin{table}[h]
    \centering

    \begin{tabularx}{\textwidth}{|X|X|}\hline

        \textbf{Casos de Uso}&  \textbf{Descripción}\\\hline
        Iniciar aplicación&El usuario inicia la aplicación para simular la carrera entre los seis vehículos.\\ \hline

    \end{tabularx}
  
\end{table}

\section{Historias de Usuario}

\justify
A continuación se describen las historias de usuario principales de la aplicación:

\begin{table}[h]

    \centering

    \begin{tabularx}{\textwidth}{|X|X|X|X|}\hline

        \textbf{ID}&  \textbf{Como un...}& \textbf{Quiero...}& \textbf{Para...}\\\hline

        HU-1&Usuario&Iniciar la aplicación&Simular el funcionamiento de una carrera entre seis coches.\\ \hline

    \end{tabularx}

\end{table}

\newpage

\section{Estructura del Proyecto}

\justify
La estructura del proyecto está organizada de la siguiente manera:

\begin{verbatim}
    
CarreraDeCochesInterfaz/
|-- DOCUMENTACIÓN/
|   |-- images/
|   |    |-- css_icon.png
|   |    |-- interfaz_1.png
|   |    |-- interfaz_2.png
|   |    |-- interfaz_3.png
|   |    |-- java_icon.png
|   |    |-- javafx_icon.png
|   |    |-- vscode_icon.png
|   |-- DOCUMENTACIÓN CARRERA DE COCHES - ADRIANO.tex
|   |-- DOCUMENTACIÓN CARRERA DE COCHES - ADRIANO.pdf
|-- src/main
|    |-- java/
|    |    |-- com/akadoblee/carreradecochesinterfaz/
|    |    |    |-- App.java
|    |    |    |-- Coche.java
|    |    |    |-- Carrera.java
|    |-- resources/
|    |    |-- carrera.fxml
|    |    |-- styles.css

\end{verbatim}

\newpage

\section{Clases Principales}

\subsection{Clase App}

\subsubsection{Descripción}

\justify
La clase \textbf{App} es la clase principal que inicia la aplicación JavaFX.

\subsubsection{Atributos}

\justify
No cuenta con atributos propios, pero maneja referencias a la interfaz gráfica.

\subsubsection{Métodos}

\justify
Los métodos de la clase \textbf{App} son los siguientes:

\begin{itemize}

    \item \textbf{start(Stage stage)}: Método que configura y muestra la ventana principal de la aplicación.
    \item \textbf{main(String[] args)}: Método principal que lanza la aplicación JavaFX.

\end{itemize}

\subsection{Clase Coche}

\subsubsection{Descripción}

\justify
La clase \textbf{Coche} representa a cada uno de los vehículos que participan en la carrera. Cada coche es un hilo independiente que simula su movimiento hasta llegar a la meta.

\subsubsection{Atributos}

\justify
Los atributos de la clase \textbf{Coche} son los siguientes:

\begin{itemize}

    \item \texttt{String} \textbf{nombre}: Nombre del coche.
    \item \texttt{int} \textbf{distanciaRecorrida}: Distancia que el coche ha recorrido hasta ese momento.
    \item \texttt{int} \textbf{distanciaTotal}: Distancia total que debe recorrer el coche para llgar a la meta.
    \item \texttt{int} \textbf{velocidadMaxima}: Velocidad máxima que puede alcanzar el coche.
    \item \texttt{Carrera} \textbf{carrera}: Referencia a la carrera en la que participa el coche.
    
\end{itemize}

\subsubsection{Métodos}

\justify
Los métodos de la clase \textbf{Coche} son los siguientes:

\begin{itemize}

    \item \textbf{run()}: Método principal del hilo que simula el movimiento del coche.
    \item \textbf{getNombre()}: Devuelve el nombre del coche.
    \item \textbf{getDistanciaRecorrida()}: Devuelve la distancia recorrida por el coche.
    \item \textbf{getDistanciaTotal()}: Devuelve la distancia que debe de recorrer el coche para finalizar la carrera
    
\end{itemize}

\subsection{Clase Carrera}

\subsubsection{Descripción}

\justify
La clase \textbf{Carrera} gestiona la lógica de la carrera, incluyendo la creación de coches, el seguimiento de su progreso y la actualización de la interfaz gráfica.

\subsubsection{Atributos}

\justify
Los atributos de la clase \textbf{Carrera} son los siguientes:

\begin{itemize}

    \item \texttt{Map<String, ProgressBar>} \textbf{barras}: Representa cada una de las barras de progreso asociadas a cada uno de los coches.
    \item \texttt{Map<String, Label>} \textbf{etiquetasDistancia}: Muestra la distancia recorrida por cada coche en metros.
    \item \texttt{List<String>} \textbf{clasificacion}: Lista que almacena el orden de llegada de los coches a la meta.
    \item \texttt{int} \textbf{DISTANCIA\_TOTAL}: Distancia total que deben recorrer los coches para llegar a la meta.

\end{itemize}

\newpage

\subsubsection{Métodos}

\justify
Los métodos de la clase \textbf{Carrera} son los siguientes:

\begin{itemize}

    \item \textbf{onStartButton()}: Crea los objetos de la clase Coche y los inicia como hilos independientes para comenzar la carrera.
    \item \textbf{añadirCarta()}: Añade una carta por cada coche en la interfaz gráfica.
    \item \textbf{actualizarProgreso()}: Actualiza las barras de progreso y las etiquetas de distancia en la interfaz gráfica.
    \item \textbf{registrarLlegada()}: Registra la llegada de un coche a la meta y actualiza la clasificación.
    \item \textbf{getClasificacion()}: Devuelve la lista con la clasificación final de los coches.

\end{itemize}

\section{Interfaz}

\justify
La interfaz gráfica de la aplicación consta de tres secciones:

\begin{itemize}

    \item \textbf{Sección Superior}: Se compone del título de la aplicación.
    \item \textbf{Sección Central}: Se compone de una sección izquierda que al principio está vacía pero posteriormente se añaden los coches en ella, y una sección derecha en la que hay un panel que muestra la clasificación de los coches conforme van llegando a la meta.
    \item \textbf{Sección Inferior}: Incluye el botón para iniciar la simulación.

\end{itemize}

\begin{figure}[h]

    \centering
    \includegraphics[width=12cm]{images/interfaz_1.png}

\end{figure}

\newpage

\justify
Una vez iniciada la carrera, la interfaz muestra el progreso de cada coche mediante barras de progreso y etiquetas que indican la distancia recorrida. \newline

\begin{figure}[h]

    \centering
    \includegraphics[width=12cm]{images/interfaz_2.png}

\end{figure}

\justify
Al finalizar la carrera, la interfaz muestra la clasificación final de los coches en el panel derecho y la carrera puede ser iniciada de nuevo pulsando nuevamente sobre le botón. \newline

\begin{figure}[h]

    \centering
    \includegraphics[width=12cm]{images/interfaz_3.png}

\end{figure}

\end{document}